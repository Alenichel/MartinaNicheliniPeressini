\documentclass{article}
\usepackage[utf8]{inputenc}
\usepackage{enumitem}


\title{Progetto Software Engineering 2}
\author{Alessandro Nichelini, Stefano Martina, Francesco Peressini}
\date{October 2018}

\begin{document}

\maketitle

\section{INTRODUCTION}

\subsection{Purpose}

TrackMe is a company aiming to support interactions between users, who like keeping track of the health status and their activities, and third parties which can use data and enhance their value.

TrackMe app is composed of three main important service:
\begin{itemize}
\item Data4Help: basic support for health data.
\item AutomatedSOS: add support for SOS services to elderly people. 
\item Track4Run: add support for running events.
\end{itemize}

Each of them will have as targets:
\begin{itemize}
\item Individuals users
\item Third parties.
\end{itemize}
Since each service serves both the targets, the project consist of a platform split in an iOS application and an API endpoint to sell and distribute data to third parties.

The system allows individual users to add and handle data on the app and to third parties to request and have access to these data.

In particular, a user of the application is able to keep track of his/her position during the activities but also during the normal day-life. The users’ data are used to monitor the heartbeat and eventually to send an AutomatedSOS.
Furthermore, trackMe, is able to track all the athletes (that are using the application), during a run previously organized. Indeed TrackMe’s app provide a section to setup a group run , specifying the path and other useful information.

\subsection{Scope}

\begin{enumerate}[label={[G\arabic*]}]
\item Allow users to add data to TrackMe manually or let the application collect them automatically.
\item Allow users to have access to an overview page of their data.
\item Allow users to permit/deny third parties to access their data.
\item Allow users to monitor her/his health parameters.
\item Allow users to monitor her/his performance during run workouts.
\item Send to user’s location emergency services in case of low vital parameters. 
\item Guarantee very efficient time reaction in case of low vital parameters detection.
\item Allow a user to create an event defining run path and other event settings.
\item Allow users to en-roll to events.
\item Allow spectators to follow participants’ live position during events.


\item Allow third parties to request users to provide their personal informations.
\item Allow third parties to request TrackMe to provide anonymised aggregated data.
\item Allow third parties to subscribe to data feeds.
\item Allow third parties to retrieve data from TrackMe through and URL endpoint (API). 
\end{enumerate}


\subsection{Definitions, Acronyms, Abbreviations}
\subsubsection{Definitions}
\subsubsection{Acronyms}
\begin{itemize}
\item API: Application Programming Interface
\item URL: Uniform Resource Locator
\item RASD:  Requirement Analysis and Specification Document 
\end{itemize}
\subsubsection{Abbreviations}

\subsection{Revision history}
\subsection{Reference Documents}
\subsection{Document Structure}

\section{OVERALL DESCRIPTION}

\subsection{Product perspective}
\subsection{Product functions}
\subsection{User characteristics}
\subsection{Assumptions, dependencies and constraints}

\section{SPECIFIC REQUIREMENTS}
\subsection{External Interface Requirements}
		\begin{itemize}
			\item User Interfaces
			\item Hardware Interfaces
			\item Software Interfaces
			\item Communication Interfaces
		\end{itemize}
\subsection{Functional Requirements}	
\subsection{Performance Requirements}
\subsection{Design Constraints}
		\begin{itemize}
			\item Standards compliance
			\item Hardware limitations
			\item Any other constraint
		\end{itemize}
\subsection{Software System Attributes}	
		\begin{itemize}
			\item Reliability
			\item Availability
			\item Security
			\item Maintainability
			\item Portability
		\end{itemize}


\section{FORMAL ANALYSIS USING ALLOY}

\section{EFFORT SPENT}

\section{REFERENCES}

\end{document}



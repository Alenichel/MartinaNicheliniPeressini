\documentclass{article}
\usepackage[utf8]{inputenc}
\usepackage{enumitem}
\usepackage{nameref}

\title{Software Engineering 2 Project}
\author{Stefano Martina, Alessandro Nichelini, Francesco Peressini}

\begin{document}

\maketitle
\newpage

\tableofcontents
\newpage

\section{Introduction}

\subsection{Purpose}

TrackMe is a company aiming to support interactions between users, 
who like keeping track of the health status and their activities, and
third parties which can use data and enhance their value.

TrackMe app is composed of three main important service:
\begin{itemize}
\item Data4Help: basic support for health data;
\item AutomatedSOS: add support for SOS services to elderly people;
\item Track4Run: add support for running events.
\end{itemize}

Each of them will have as targets:
\begin{itemize}
\item Individuals users
\item Third parties.
\end{itemize}
Since each service serves both the targets, the project consists of 
platform divided in a mobile application and a web-based interface to
serve third-parties.

The system allows individual users to add and handle data on the app
and to third parties to request and have access to these data.

In particular, a user of the application is able to keep track of 
his/her position during the activities but also during the normal 
day-life. The users’ data are used to monitor the heartbeat and 
eventually to send an AutomatedSOS.
Furthermore, TrackMe, is able to track all the athletes (that are 
using the application), during a run previously organised. 
Indeed TrackMe’s app provide a section to setup a group run, 
specifying the path and other useful information.

\subsection{Scope}
\subsubsection{Description of the problem}
Nowadays a lot of persons track their activities with smartphone or
smartwatch. For this reason TrackMe provide a new complete user 
experience allowing all the user to read briefly all the information
about all activities’ history.
It’s also provided a service to organise a group run, during which it’s possible to monitor all the athletes information.
Furthermore all the users are monitored and, in case of some trouble, an SOS will be launched.

\subsubsection{World Phenomena}
\begin{itemize}
	\item \textit{General user’s health condition}: the machine doesn’t know the information about possible user’s disease.
	\item \textit{First aid services status}: the machine doesn’t know the actual first aid services status.
	\item \textit{Overall third parties knowledge status}: the machine doesn’t know which informations third parties already have about users.
\end{itemize}

\subsubsection{Machine Phenomena}
\begin{itemize}
	\item \textit{Third-parties registration};
	\item \textit{User's registration};
	\item \textit{Data anonymisation}.
\end{itemize}

\subsubsection{Shared Phenomena}
\begin{itemize}
	\item \textit{Vital parameters}: the machine can read vital parameters of the user such as BPM.
	\item \textit{User's location}: the machine knows or can read actual and past user’s location.
\end{itemize}

\subsubsection{Goals} \label{goals}
\begin{enumerate}[label={\textbf{[G\arabic*]}}]
\item Users can be recognised by their credentials.
\item Allow users to keep track of their health data.
\item Allow users to have access to an overview of their data, including health parameters and performed activities. \label{g3}
\item Allow users to manage their data access policy.
\item Allow users to monitor their performance during run workouts.
\item Each time vital signs go below a threshold value, first aid services have to be notified.
\item Allow users to organise running events.
		\begin{enumerate}[label={[G\arabic{enumi}.\arabic*]}]
    		\item Allow users to create running events.
    		\item Allow users to en-roll to events.
    		\item Allow spectators to follow participants’ live position during events.
  		\end{enumerate}
\item Allow third parties to access data:
		\begin{enumerate}[label={[G\arabic{enumi}.\arabic*]}]
    		\item Allow third parties to require access to specific user data.
    		\item Allow third parties to retrieve anonymised aggregated data.
    		\item Allow third parties to subscribe to da
  		\end{enumerate}

\end{enumerate}

\subsection{Definitions, Acronyms, Abbreviations}

\subsubsection{Definitions}
\begin{itemize}
	\item 
\end{itemize}

\subsubsection{Acronyms}
\begin{itemize}
\item API: Application Programming Interface;
\item ASOS: AutomatedSOS;
\item BPM: Beats Per Minutes;
\item D4H: Data4Help;
\item RASD: Requirement Analysis and Specification Document;
\item T4R: Track4Run.
\end{itemize}

\subsubsection{Abbreviations}
\begin{itemize}
		\item \begin{math}[Gn]\end{math}: n-th goal
		\item \begin{math}[Dn]\end{math}: n-th domain assumption 
		\item \begin{math}[Rn]\end{math}: n-th functional requirement
\end{itemize}

\subsection{Revision history}
\subsection{Reference Documents}
\subsection{Document Structure}

\section{Overall description}

\subsection{Product perspective}

\subsection{Product functions}
The software developed by TrackMe included three different services:
Data4Help supports user's data acquiring through smartwatches or 
similar devices, AutomatedSOS, a personalised and non-intrusive SOS 
service for elderly people and Track4Run, a service to track athletes 
participating to a run.
Data4Help is also a service available for third-parties: in fact, 
organisations can request data to TrackMe and collect them for 
pursuing their objectives.
Data acquisition can be performed in two different way: directly to a
single user or to a groups of individuals.
Concerning the single user case, companies, through a social security
code, ask permission to access the corresponding information. 
In the other case, organisations can request, directly to TrackMe,
data of group of individuals with particular proprieties (e.g. users 
between 20 and 30 years old or living in a certain district).
In the latter, TrackMe provides data only if it can anonymised them
correctly.
AutomatedSOS it’s great for those elderly people who want to monitor 
their health status and have the support of an ambulance in case their
vital parameters are under certain thresholds.
Track4Run it’s a service dedicated to the runners: it offers the 
possibility to organise a run between other runners, define the path 
and the duration, allowing the invited ones to en-roll. In addition, 
Track4Run users who do not partecipate to the run can see live time 
on a map the position of all the runners. 
Finally, Track4Run supports the synchronisation of data with 
Data4Help.

All the goals presented in section \ref{goals} are going to be 
implemented. Here we describe deeply the requirements needed to
implement application's functions.
	
\begin{enumerate}[label={[R\arabic*]}]
    	\item Users can create an account with credentials.
    	\item Credentials can be retrievable also if forgotten/lost.
    	\item Users can log manually or automatically their data.
    	\item Users have to be able to accept/deny access to single data access request.
    	\item Users have to be able to see current data policies and to change them.
    	\item The machine has to be able to read health and position data.
		\item The machine has to be able to recognise below threshold parameters.
		\item The machine has to be able to communicate with third parties.
		\item The machine has to be able to recognise data fragmentation level 
		\item The machine has to be able to store users’ data. 


\end{enumerate}

\subsection{Users characteristics}
\textbf{Users}: any person that will use the service to keep track of health data, positioning or automated sos during any kind of activities (walking or running during a loner run or an organised event).\\\\
\textbf{Third parties}: Organisation or other parties interested in the acquisition for their usage of the data provided by the user activities.

\subsection{Assumptions, dependencies and constraints}

\subsubsection{Domain assumptions}

\begin{enumerate}[label={[D\arabic*]}]
    	\item Data manually logged by users well describes reality.
    	\item First aid services are ready to handle emergency notifications.
    	\item Data inserted by users are truthful 
    		\begin{enumerate}[label={[D\arabic{enumi}.\arabic*]}]
    			\item Data are up to date.
    			\item Data are correctly formatted.
  			\end{enumerate}
  		
  		\item Running paths proposed by users are well formed (e.g: legit by law).
  		\item External services are reliable.
\end{enumerate}

\section{Specific requirements}
\subsection{External Interface Requirements}
		\begin{itemize}
			\item User Interfaces
			\item Hardware Interfaces
			\item Software Interfaces
			\item Communication Interfaces
		\end{itemize}
\subsection{Functional Requirements}	

\subsection{Performance Requirements}
\subsection{Design Constraints}
		\begin{itemize}
			\item Standards compliance
			\item Hardware limitations
			\item Any other constraint
		\end{itemize}
\subsection{Software System Attributes}	
		\begin{itemize}
			\item Reliability
			\item Availability
			\item Security
			\item Maintainability
			\item Portability
		\end{itemize}
		
\subsection{Scenarios}

\subsubsection{Scenario 1}
Tim has to sustain an important medical exam aimed to discover if he
suffers or not of tachycardia. He has available the latest model of 
smartwatch that supports the EGC monitor so his doctor suggests to 
use Data4Help to monitor for 24/48 hours his heart rate. Data4Help 
allows Tim to just use his smartwatch, instead of the classic Dynamic 
ECG machine, for registering his heart’s data in the app so that his
doctor, after committing a request to Tim through his fiscal code, 
can examine the results after few hours by the end of the test and 
give to Tim a response in a very short time. 

\subsubsection{Scenario 2}
Fitness \& More is a brand new sport center situated near a very big 
residential area with many educational institutions from elementary to 
high schools. The centre offers swimming and tennis lessons with
expert instructors and also a well supplied gym with personal coaches.
Fitness \& More asks to TrackMe the access to weight and height data 
of kids between 6 and 19 year old who live in the above mentioned area 
to make an analysis of the presence of overweighted individuals and so
sponsor its sport center. 

\subsubsection{Scenario 3}
The next PolimiRun will be held on the 11th of November in Lecco.
Polisport, the sport organisation of the Politecnico di Milano, 
decided to arrange a few workouts for the runners who are attending 
the competition and so they suggest to use Track4Run to manage the 
path of the workouts around the city of Milan and also to let runners
en-roll the training sessions. 

\section{Formal analysis using Alloy}

\section{Effort spent}

\section{References}

\end{document}
\documentclass[DD.tex]{subfiles}
\begin{document}

\section{Introduction}

\subsection{Purpose}
The purpose of this Design Document is to give a functional description of the Data4Help application. The focus is on how the whole system is implemented, with particular attention to the components and its architecture. \newline
While in the RASD document the description is presented at a high level, in the DD all the relevant components and their interfaces are explained in details, accompanied with the related diagrams.
Algorithms of the most important features of the application and Implementation, Integration and Test Plan are also presented in this document. 

\subsection{Scope}
The main scope of the Data4Help project is to provide a platform where the user can consult his/her health and location data and share these data with third-parties so that the latter can increase their business value. \newline
The application also offers two additional services: AutomatedSOS, a constant vital parameters monitoring service, thought for elderly people, to detect anomalies and, in case of emergency, call an ambulance directly to the location of the costumer, and Track4Run, a service mainly thought for runners to create and join running events and live tracking on a map participants of the runs. \newline
Both AutomatedSOS and Track4Run exploit the features offered by Data4Help.

\subsection{Definitions, Acronyms, Abbreviations}

\subsubsection{Definitions}

\subsubsection{Acronyms}
\begin{itemize}
	\item API: Application Programming Interface;
	\item BPM: Beats Per Minutes;
	\item DD: Design Document;
	\item RASD: Requirement Analysis Specification Document;
	\item JSON: JavaScript Object Notation.
\end{itemize}

\subsubsection{Abbreviations}
\begin{itemize}
	\item \begin{math}[Gn]\end{math}: n-th goal
	\item \begin{math}[Dn]\end{math}: n-th domain assumption 
	\item \begin{math}[Rn]\end{math}: n-th functional requirement
\end{itemize}

\subsection{Revision history}
\begin{itemize}
		\item 1.0.0 Initial version (10/12/2018)
\end{itemize}

\subsection{Reference Documents} 
\begin{itemize}
		\item RASD document previously delivered 
\end{itemize}

\subsection{Document Structure}

\newpage

\end{document}